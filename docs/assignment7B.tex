\documentclass[12pt]{article}

\pagestyle{empty}

\begin{document}

\section*{Assignment Set for Laboratory 7b: Optional Lab}

EOSC 511/ATSC 506: 

\begin{enumerate}
\item Hand-in an answer to question 7b from the lab itself.

\item You will use the interactive1.py code in numlabs/lab7 for this
  question.  If your experience with the Coriolis force is minimal,
  you can chose the ``small'' option.  See the doc string for
  reasonable parameters.

The code solves the high water level in the center (similar to rain.py)
in two-dimensions with periodic boundary conditions on a flat bottom.
The depth is set in the functions find\_depth*.  Use
grid-C and edit the find\_depth3 function.  
\begin{enumerate}
\item Choose an interesting but smooth topography (remembering that
  the domain is periodic in both space dimensions).  Implement it in
  find\_depth3 correctly given the grid-C staggering.  
\item Run your new code. Discuss any other changes you make to the code. You may want to change what and when it plots.
\item Explain the differences that the bottom topography makes.
\end{enumerate}

\end{enumerate}

\end{document}

%%% Local Variables:
%%% mode: latex
%%% TeX-master: t
%%% End:
